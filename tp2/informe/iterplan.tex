\begin{enumerate}
	\item $\bullet$ $\rightarrow$ \textit{Caso de uso}.
	\item $\circ$ $\rightarrow$ \textit{Tarea}.
\end{enumerate}

\begin{center}
	\begin{tabular}{| c | p{10cm} | c |}
		\hline
		\textbf{Primera iteración}& $\circ$ Obteniendo información definida de Twitter. & 85 hs  \\
								  & $\bullet$ Almacenando datos obtenidos de distintos sitios de internet. & 70 hs \\
								  & $\bullet$ Si realizo una consulta por un producto A, obtengo ofertas de este producto. & 32 hs \\
								  & $\bullet$ Mostrando publicidades en la aplicación. & 40 hs \\
								  & $\circ$ Establecer una primera interfaz de fácil implementación. & 39 hs \\
								  & $\circ$ Diseño conceptual del sistema. & 5 hs \\ 
								  & $\circ$ Realización de WBS. & 14 hs \\
								  & $\circ$ Primera revisión del caso de negocio, objetivos y requerimientos. & 12 hs \\
								  & $\circ$ Análisis de riesgos. & 2 hs \\
								  & $\circ$ Definición y priorización de casos de uso, documentando los tiempos esperados para los casos de uso más prioritarios. & 12 hs \\
								  & $\circ$ Definir junto a los stakholder los atributos de calidad del sistema. & 6 hs \\
								  & $\circ$ Diseño de la Arquitectura, teniendo en cuenta las restricciones a las tecnologías impuestas por los stakholder. & 30 hs \\

		\hline
		\textbf{Segunda iteración}& $\bullet$ El usuario se autentifica con el sistema. & 80 hs \\
								  & $\bullet$ Un usuario autentificado puede votar la validez de una oferta. &  50 hs \\
								  & $\bullet$ Si realizo una consulta por un producto A y soy un usuario autentificado, las ofertas recibidas se priorizan acorde a mis preferencias de confianza. & 200 hs \\
								  & $\bullet$ ABM de rubros habilitados. & 50 hs \\
								  & $\bullet$ ABM de productos en un rubro. & 50 hs \\
								  & $\bullet$ Si realizo una consulta por un producto A y este no esta, se le informa al usuario. & 25 hs \\

		\hline
		\textbf{Tercera iteración}& $\bullet$ Se realizó una consulta por un producto, y el dispositivo sin tener conexión, logra responder la consulta de alguna forma medianamente satisfactoria. & 75 hs \\
								  & $\bullet$ Detectando ofertas falsas. & 350 hs\\
								  & $\bullet$ Siendo martes se publica un informe de ofertas falsas. & 50 hs \\
								  & $\bullet$ Se prepara un informe con las estadisticas de ofertas detectadas como falsas. & 50 hs\\

		\hline
		\textbf{Cuarta iteración}& $\bullet$ El usuario consulta precios a través de una interfaz amigable en su celular. & 250 hs \\
								 & $\bullet$ Si realizo una consulta por un producto A, y se considera que puede sustituirse por B, tambien se muestran ofertas de B. & 100 hs \\
								 & $\bullet$ Si realizo una consulta por un producto A, que se considera asociado con B, tambien se muestran ofertas de B. & 100 hs \\

		\hline
		\textbf{Quinta iteración}& $\bullet$ Almacenando datos obtenidos de distintos sitios de internet. & 350 hs \\
								 & $\bullet$ El usuario consulta precios a través de una interfaz amigable en su computadora. & 250 hs \\
								 & $\bullet$ Se consulta información a travéz de el API publica. & 50 hs \\

		\hline
	\end{tabular}
\end{center}
