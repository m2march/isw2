\documentclass[10pt, a4paper]{article}

\usepackage[paper=a4paper, left=1.5cm, right=1.5cm, bottom=1.5cm, top=1.5cm]{geometry}
\usepackage[utf8]{inputenc}
\usepackage[spanish]{babel}
\usepackage{graphicx}
\usepackage{multicol}
\usepackage[usenames,dvipsnames]{color}
\usepackage{amsmath}
\usepackage{verbatim}
\usepackage{footnote}
\usepackage{float}
\usepackage{amsfonts}
\usepackage{hyperref}
\usepackage{framed}
\usepackage{pdflscape}

\usepackage{pdfpages}

\usepackage{caratula}

\materia{Ingeniería de Software II}

\titulo{Trabajo Práctico 2}
\subtitulo{Developing \emph{in-the-large} - Planificación}
\grupo{Grupo 5 - \emph{El nene está bien}}

\integrante{Martín Alejandro Miguel}{181/09}{m2.march@gmail.com}
\integrante{Iván Postolski}{216/09}{ivan.postolski@gmail.com}
\integrante{Juan Manuel Martinez Caamaño}{276/09}{jmartinezcaamao@gmail.com}
\integrante{Matías Incem}{396/09}{matias.incem@gmail.com}
\integrante{Pablo Gauna}{334/09}{gaunapablo@gmail.com}


\begin{document}

\maketitle
\tableofcontents
\newpage

\section{Introducción}

El presente informe constituye la planficación del proyecto \textbf{Twitteando para ahorrar} (TPA), que conforma una extensión del proyecto \textbf{Precio Justo} a partir de los intereses de parte de autoridades gubernamentales. Para este nuevo proyecto la metodología de trabajo utilizada se basará en la metodología \emph{RUP}, a causa del fuerte cambio en la escala de la aplicación respecto del trabajo anterior.  

\subsection{Caso de negocio}

El caso de negocio se modificó respecto del trabajo anterior principalmente en la escala del software a desarrollar. Esto incluye tanto un aumento en la base de usuarios de nuestra aplicación como en las funcionalidades a proveer. A continuación enunciamos el caso de negocio como una serie de objetivos a lograr:

\begin{itemize}
\item La información de precios no solo será obtenida de \emph{twitter}, sino que también de otras redes sociales (\emph{facebook}, \emph{pinterest}) y distintos sitios web (sitios de remate, sitios de supermercados, sitios de ofertas).
\item La aplicación movil que accederá a los datos recopilados debe tener usabilidad accesible a toda la población, lo que implica una interfáz, limpia y amigable.
\item La información recopilada también será provista de forma pública para otras aplicaciones, generando todo un nuevo lado de usabilidad a nuestra aplicación.
\item La presentación de información contará con más inteligencia, mostrando no solo productos que el usuario declaró como interesnates, sino también información de otros productos que se encuentren relacionados.
\item El uso de la aplicación será personalizable, dándole la oportunidad al usuario de definir niveles de confianza de las ofertas según su origen.
\item La detección de información falsa o sospechosa deberá ser muy robusta. Además, los detalles de la detección deberá ser visible.
\item Se continuará trabajando con una serie de \emph{productos} aceptados por la aplicación, pero estos se agruparán bajo el concepto de \emph{rubros}. La administración de productos y rubros tendrá como fin mejorar el manejo de información internamiente en el sistema.
\end{itemize}

\subsection{Detalles y restricciones}

Para el desarrollo de la aplicación deberemos tener en cuenta los siguientes detalles:

\begin{itemize}
\item No contaremos con servidores potentes para el procesamiento de datos, pero si con la cantidad que nos sea necesaria de equipos de escritorio.
\item Se nos exige que los datos se guarden en una base de datos \textbf{NOSQL}
\item Se nos exige que los servicios de provean desde \emph{la nube} utilizando servicios de \textbf{cloud computing}
\item Por limitaciones de presupuesto la aplicación debe ser capaz de generar ingresos en el corto plazo
\item Se nos sugiere el servicio SpamBust para lograr la detección de ofertas falsas. No obstante el servicio es pago y deberá ser reemplazado en la brevedad.
\end{itemize}

\subsection{Diccionario de términos}

\begin{itemize}
\itemsep1pt\parskip0pt\parsep0pt
\item
  \textbf{Oferta}: Información sobre donde comprar un producto X y a qué
  precio.
\item
  \textbf{Usuario autenticado}: Usuario del cual el sistema tiene
  seguridad quién es.
\end{itemize}


\section{Plan de proyecto}



\end{document}
