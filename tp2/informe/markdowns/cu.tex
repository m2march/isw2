\begin{itemize}
\item
  Obteniendo informacion de internet \textbf{Descripción}: El sistema
  colecta la información de los distintos medios, la procesa, y la
  almacena para luego ser provista a los usuarios.
\item
  Se consulta información a travéz de el API publica.
  \textbf{Descripción}: Clientes externos pueden consultar a nuestro
  sistema por precios que recopilamos de distintos medios a través de un
  servicio público (API) ofrecido por nuestro sistema.
\item
  El usuario consulta precios a través de una interfaz amigable
  \textbf{Descripción}: El usuario accede a una aplicación de celular
  propia de \emph{twitteando para ahorrar} a través de la cual puede
  consultar por precios para distintos productos.
\item
  Se realizó una consulta por un producto, y el dispositivo sin tener
  conexión, logra responder la consulta de alguna forma medianamente
  satisfactoria. \textbf{Descripción}: El usuario accede a la aplicación
  de celular \emph{twitteando para ahorrar} sin tener conectividad a
  internet y recibe precios de los productos deseados y los relacionados
  a estos. La información provista al usuario podría estar limitada
  respecto de lo que vería si tuviera conectividad, pero esto no debería
  ser notado por el mismo.
\item
  ABM de rubros habilitados. \textbf{Descripción}: Ciertos usuarios
  particulares del sistema pueden acceder al mismo para agregar nuevos
  rubros o modificar o borrar existentes.
\item
  ABM de productos en un rubro. \textbf{Descripción}: Ciertos usuarios
  particulares del sistema pueden acceder al mismo para agregar nuevos
  productosi o modificar o borrar existentes. También pueden redefinir
  la pertenencia de un producto a uno o más rubros.
\item
  Si realizo una consulta por un producto A, obtengo ofertas de este
  producto. \textbf{Descripción}: El usuario consulta por un producto A
  dentro de los habilitados en algún rubro y recibe información de donde
  comprarlo y a qué precio.
\item
  Si realizo una consulta por un producto A y este no está se le informa
  al usuario. \textbf{Descripción}: El usuario consulta por un producto
  A que no está habilitado en ningún rubro y es informado que el sistema
  no posee información sobre donde comprar el mismo.
\item
  Si realizo una consulta por un producto A, y se considera que puede
  sustituirse por B, tambien se muestran ofertas de B.
  \textbf{Descripción}: El usuario consulta por un producto A dentro de
  los habilitados en algún rubro y se definió que puede sustituirse por
  el producto B, luego el usuario recibe información de donde comprar A
  y donde comprar B y a qué precio.
\item
  Si realizo una consulta por un producto A, que se considera asociado
  con B, tambien se muestran ofertas de B. \textbf{Descripción}: El
  usuario consulta por un producto A dentro de los habilitados en algún
  rubro y se definió que está asociado con el producto B, luego el
  usuario recibe información de donde comprar A y donde comprar B y a
  qué precio.
\item
  Si realizo una consulta por un producto A y soy un usuario
  autentificado, las ofertas recibidas se priorizan acorde a mis
  preferencias de confianza. \textbf{Descripción}: Dentro de las ofertas
  relacionadas al producto A que conoce el sistema, se mostrarán primero
  aquellas cuya fuente yo haya declarado de mayor confianza, luego las
  de fuentes con menor confianza y no se mostrará ninguna oferta cuya
  fuente declaré como no confiable.
\item
  Mostrando publicidades \textbf{Descripción}: Cuando el usuario utiliza
  la aplicación movil visualiza, aparte de los resultados de su
  consulta, propaganda de los spónsores de \emph{twitteando para
  ahorrar}.
\item
  Detectando ofertas falsas \textbf{Descripción}: Al recopilar datos de
  precios en internet, el sistema es capaz de detectar si la información
  es sospechosa y marcarla como tal, para futura revisión. Además el
  sistema recopila todas las evidencias encontradas para sospechar de
  los datos.
\item
  Siendo martes se publica un informe de ofertas falsas
  \textbf{Descripción}: Cada martes el sistema arma y publica un informe
  con los productos sobre los cuales se encontraron precios dudosos
  junto con la evidencia que genera la sospecha. Este informe debe estar
  disponible para revisión por usuarios externos selectos.
\item
  Se prepara un informe con las estadisticas de ofertas detectadas como
  falsas. \textbf{Descripción}: Al mismo tiempo que el usuario comienza
  a ingresar una consulta en la aplicación movil, la aplicación se
  anticipa a los deseos del usuario para mostrarle rápidamente precios
  de productos que podríán responder a la consulta que se está
  formulando.
\item
  El usuario se autentica con el sistema \textbf{Descripción}: El
  usuario de la aplicación movil puede utilizar alguna cuenta de un
  servicio asociado con OpenID (google, yahoo, facebook y otro) para
  autenticarse en la aplicación. A partir de ese momento la aplicación
  sabe quién es el usuario y puede utilizar la información que tiene del
  mismo para proveerle funcionalidades más avanzadas.
\item
  Un usuario autentificado puede votar por la validez de una oferta.
  \textbf{Descripción}: Un usuario ya autenticado en el sistema elije
  una oferta y la marca como válida o inválida. Esto afecta la
  reputación del usuario o fuente que dio origen a la oferta para
  facilitar la detección de ofertas sospechosas.
\end{itemize}
