\subsubsection{Riesgo1:}

\begin{itemize}
\itemsep1pt\parskip0pt\parsep0pt
\item
  \textbf{Descripción}: Desconocimiento de las tecnologías a usar:
  OpenID, base de datos nosql, webmining, sms, cloud computing, etc. que
  podría llegar a ocasionar retrasos inesperados en el proyecto.
\item
  \textbf{Probabilidad}: Media
\item
  \textbf{Impacto}: Bajo
\item
  \textbf{Exposición}: Baja
\item
  \textbf{Mitigación}: Dedicar horas de aprendizaje, lo antes posible,
  para cada tecnología y saber de antemano los problemas que podrían
  suceder.
\item
  \textbf{Plan de contingencia}: Contratar gente especializada en las
  tecnologías.
\end{itemize}

\subsubsection{Riesgo2:}

\begin{itemize}
\itemsep1pt\parskip0pt\parsep0pt
\item
  \textbf{Descripción}: Por falta de definiciones claras y consisas el
  código desarrollado no cumple con las expectativas de los stakeholders
  y se vuelve necesario modificarlo. Se consideran particularmente
  riesgosas las definiciones en: criterios de sustitución y asociación,
  información a recuperar, prioridad de los usuarios por la información.
\item
  \textbf{Probabilidad}: Alta
\item
  \textbf{Impacto}: Alto
\item
  \textbf{Exposición}: Alta
\item
  \textbf{Mitigación}: Acordar fechas límites para la presentación de
  documentación definiendo mejor cada caso en particular. Se aislaran y
  se implementaran los modulos correspondientes a estas caracteristicas
  de forma que su modificación resulte lo mas rapida posible.
  (Modificabilidad)
\item
  \textbf{Plan de contingencia}:
\end{itemize}

\subsubsection{Riesgo3:}

\begin{itemize}
\itemsep1pt\parskip0pt\parsep0pt
\item
  \textbf{Descripción}: Inconvenientes con el servicio de spambust que
  nos deja sin esta funcionalidad. (E.g.: sube el precio, problemas de
  servicio por su parte).
\item
  \textbf{Probabilidad}: Baja
\item
  \textbf{Impacto}: Alto
\item
  \textbf{Exposición}: Media
\item
  \textbf{Mitigación}: Implementar, lo antes posible, un servicio propio
  que remplace el de spambust.
\item
  \textbf{Plan de contingencia}: Dejar de detectar el spam o detectarlo
  en menor medida con una implementación fácil y rápida pero no tan
  efectiva.
\end{itemize}

\subsubsection{Riesgo4:}

\begin{itemize}
\itemsep1pt\parskip0pt\parsep0pt
\item
  \textbf{Descripción}: Falta de dinero para financiar el proyecto.
\item
  \textbf{Probabilidad}: Media
\item
  \textbf{Impacto}: Alto
\item
  \textbf{Exposición}: Alta
\item
  \textbf{Mitigación}: Conseguir sponsor, lo antes posibles, para que el
  proyecto sea auto-sostenible.
\item
  \textbf{Plan de contingencia}: Resignar funcionalidades del proyecto
  que requiera mucha inversión para reducir los costos.
\end{itemize}

\subsubsection{Riesgo5:}

\begin{itemize}
\itemsep1pt\parskip0pt\parsep0pt
\item
  \textbf{Descripción}: El hardware disponible no es suficiente para
  soportar TODAS las funcionalidades a cumplir (extraer datos,
  publicarlos mediante la api, predecir los deseos del usuario,
  verificar los datos recibidos, generar información de auditoría).
\item
  \textbf{Probabilidad}: Media
\item
  \textbf{Impacto}: Alto
\item
  \textbf{Exposición}: Alta
\item
  \textbf{Mitigación}: Mas HW, mejores algoritmos.
\item
  \textbf{Plan de contingencia}: Resignar alguna de las funcionalidades.
\end{itemize}

\subsubsection{Riesgo6:}

\begin{itemize}
\itemsep1pt\parskip0pt\parsep0pt
\item
  \textbf{Descripción}: Un miembro abandona el equipo de trabajo.
\item
  \textbf{Probabilidad}: Media
\item
  \textbf{Impacto}: Medio
\item
  \textbf{Exposición}: Media
\item
  \textbf{Mitigación}: Particionado al sistema en `modulos' simples con
  una única responsabilidad y alta cohesión interna, y manteniendo la
  documentación de la arquitectura del sistema actualizada, de forma que
  al incorporar un remplazo, la curva de aprendizaje del sistema se vea
  reducida.
\item
  \textbf{Plan de contingencia}: Contratar remplazo.
\end{itemize}

\subsubsection{Riesgo7:}

\begin{itemize}
\itemsep1pt\parskip0pt\parsep0pt
\item
  \textbf{Descripción}: Debido a los tiempos acotados requerido para
  sacar al mercado el producto, los módulos podrían carecer del testing
  adecuado.
\item
  \textbf{Probabilidad}: Media
\item
  \textbf{impacto}: Bajo
\item
  \textbf{Exposición}: Baja
\item
  \textbf{Mitigación}: Mantener un mínimo nivel de testing requerido en
  cada módulo.
\item
  \textbf{Plan de contingencia}: Reparar bugs y mejorar los test del
  módulo.
\end{itemize}

\subsubsection{Riesgo8:}

\begin{itemize}
\itemsep1pt\parskip0pt\parsep0pt
\item
  \textbf{Descripción}: Los deseos del usuario sobre la aplicación no
  son los esperados por los stakeholders y es necesario cambiar
  drásticamente la funcionalidad.
\item
  \textbf{Probabilidad}: Media
\item
  \textbf{Impacto}: Media
\item
  \textbf{Exposición}: Media
\item
  \textbf{Mitigación}: Tener una versión beta de la aplicación de
  usuario para tener feedback lo antes posible y poder agregar las
  modificaciones de forma suave y sin desperdiciar tiempo de trabajo.
\item
  \textbf{Plan de contingencia}: Hacer estudios de marketing para
  comprender realmente las necesidades del usuario.
\end{itemize}
