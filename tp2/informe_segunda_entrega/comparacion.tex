\subsection{Comparación Metodologias agiles y Unified Process}

	Tanto \emph{Unified Process} como \emph{Scrum} son metodologias para el desarrollo de software, estas proponen un marco de trabajo para planificar y controlar distintas etapas en el desarrollo de software.

	Una caracteristica comun en ambas metodologias, es que siguen un proceso \emph{iterativo incremental}. Esto quiere decir que en ambas metodologias, el desarrollo del software se realiza en forma de iteraciones bien definidas. Una de las principales diferencias, encontradas durante la realización de los trabajos practicos de la materia, es que Unified Process organiza el desarrollo de software en distintas \emph{etapas} (\emph{inception, elaboration, construction} y \emph{transition}), cada etapa se diferencia de las otras en la actividad sobre la cual se pone enfasis. Sin embargo, en cada etapa se podrian realizar cada una de las posibles actividades (desarrollo, requerimientos, testing, etc...). Por otro lado, metodologias agiles como Scrum, no establecen nignuna distinción de etapas en el desarrollo.

	Esto se refleja en que para desarrollar la aplicación \emph{Precio Justo}, se partio de una escasa planeación sobre la construcción del software a desarrollar, en cambio, para \emph{Twitteando para ahorrar} se parte de un \emph{plan de iteraciones} y de una \emph{elaboración de la arquitectura} bien definida y documentada (si bien este plan y la arquitectura pueden estar sujetos a cambios).

	El desarrollo en Scrum, se centra en priorizar las funcionalidades que otorguen mas 'valor' al producto, esto se refleja en la idea de organizar las tareas a realizar en cada iteración en \emph{user stories}, una serie de historias, similares a casos de uso, contadas desde el punto de vista del usuario/product owner de la aplicación. En cambio, en Unified Process, las tareas a realizar en cada iteración estan son elegidas con el objetivo de reducir los riesgos. Para capturar las funcionalidades a implementar en cada iteración, Unified Process utiliza \emph{casos de uso}.

	De tener que escoger una metodologia para el desarrollo de software, nosotros creemos que \textbf{no} tiene sentido elegir de forma \emph{estricta} una unica metodologia. Nosotros proponemos utilizar ambas metodologias como un marco de ideas para construir una metodologia que se adapte de la mejor forma a nuestro equipo y al proyecto con el cual nos enfrentemos.
	De ambas metodologias, las caracteristicas que mas nos gustaron son:

	\begin{itemize}
		\item En este momento, Juan esta QUE-MA-DO.
	\end{itemize}

\subsection{Comparación 'programming in the small' y 'programming in the large'}

