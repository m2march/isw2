\subsection{Comparación Metodologias agiles y Unified Process}

	Tanto \emph{Unified Process} como \emph{Scrum} son metodologias para el desarrollo de software, estas proponen un marco de trabajo para planificar y controlar distintas etapas en el desarrollo de software.

	Una caracteristica comun en ambas metodologias, es que siguen un proceso \emph{iterativo incremental}. Esto quiere decir que en ambas metodologias, el desarrollo del software se realiza en forma de iteraciones bien definidas. Una de las principales diferencias, encontradas durante la realización de los trabajos practicos de la materia, es que Unified Process organiza el desarrollo de software en distintas \emph{etapas} (\emph{inception, elaboration, construction} y \emph{transition}), cada etapa se diferencia de las otras en la actividad sobre la cual se pone enfasis. Sin embargo, en cada etapa se podrian realizar cada una de las posibles actividades (desarrollo, requerimientos, testing, etc...). Por otro lado, metodologias agiles como Scrum, no establecen nignuna distinción de etapas en el desarrollo.

	Esto se refleja en que para desarrollar la aplicación \emph{Precio Justo}, se partio de una escasa planeación sobre la construcción del software a desarrollar, en cambio, para \emph{Twitteando para ahorrar} se parte de un \emph{plan de iteraciones} y de una \emph{elaboración de la arquitectura} bien definida y documentada (si bien este plan y la arquitectura pueden estar sujetos a cambios).

	El desarrollo en Scrum, se centra en priorizar las funcionalidades que otorguen mas 'valor' al producto, esto se refleja en la idea de organizar las tareas a realizar en cada iteración en \emph{user stories}, una serie de historias, similares a casos de uso, contadas desde el punto de vista del usuario/product owner de la aplicación. En cambio, en Unified Process, las tareas a realizar en cada iteración estan son elegidas con el objetivo de reducir los riesgos. Para capturar las funcionalidades a implementar en cada iteración, Unified Process utiliza \emph{casos de uso}.

	De tener que escoger una metodologia para el desarrollo de software, nosotros creemos que \textbf{no} tiene sentido elegir de forma \emph{estricta} una unica metodologia. Nosotros proponemos utilizar ambas metodologias como un marco de ideas para construir una metodologia que se adapte de la mejor forma a nuestro equipo y al proyecto con el cual nos enfrentemos.

	Una de las diferencias que encontramos entre ambas metodologias, es que Scrum pide al final de cada iteración presentar un entregable, potencialmente listo a ser usado. De esta forma, se obtiene feedback temprano de los interesados en el proyecto. En cambio en UP, no necesariamente hace falta presentar un entregable con capacidad de funcionar.

	De ambas metodologias, las caracteristicas que mas nos gustaron son:

	\begin{itemize}
		\item El desarrollo \textbf{iterativo incremental}: El metodo iterativo incremental nos permite tener 'checkpoints' bien definidos, distribuidos mas o menos uniformemente en la linea de tiempo. Esto facilita el establecer objetivos chicos, alcanzables en el corto plazo, que ayudan a mantener al equipo focalizado. Ademas ayudan a obtener feedback de los interesados tempranamente. Esto ultimo es especialmente importante para descubrir si las expectativas de los interesados eran distintas a las del equipo de desarrollo, y permite tomar acciones rapidamente, para contrarrestar esta situación.

		\item De Scrum, la idea de \textbf{al final de cada iteración presentar un entregable}, para obtener feedback temprano de los usuarios. Sin embargo, en algunos casos es necesario ser flexible con esta condicion, en especial para proyectos grandes o que tienen poca interacción con el usuario, en las primeras iteraciones del proyecto.

		\item De Unified Process, el separar el desarrollo del sistema en \textbf{etapas}. Debido a que es esperable que en las primeras iteraciones, los esfuerzos se concentren en obtener requerimientos y planificar el proyecto, y que una vez que el proyecto se encuentra mas maduro, el foco se centre en programación, testing e integración.

		\item Las ideas de organizar el desarrollo en funcionalidades, representadas con \emph{user stories}, y decidir cuando implementar cada funcionalidad de acuerdo a una combinación de \textbf{business values} y minimización de \textbf{riesgos}. De esta forma, se intenta satisfacer cuanto antes los intereses de los stakeholders, y se intenta obtener feedback de ellos cuanto antes y al mismo tiempo, se tienen en cuenta los riesgos, que tienden a afectar fuertemente la arquitectura del sistema y de no tenerse en cuenta de forma temprana, podrian poner en riesgo el proyecto, o los tiempos planeados para completarlo.

		\item La elaboración detallada de la \textbf{arquitectura} como uno de los puntos centrales de la metodologia. Para, desde el inicio del proyecto, planificar el software para poder contrarrestar \emph{riesgos} y satisfacer los \emph{atributos de calidad} que restringen al proyecto. Ademas, el desarrollo de la arquitectura, suele dividir al sistema, en subsistemas mas pequeños, de los cuales es esperable que uno o dos desarrolladores puedan completar. Esto ayuda a la hora de asignar trabajo a los distintos miembros del equipo de desarrollo.

		\item De las metodologias agiles, uno de sus pautas claves, el \textbf{aceptar cambios} en los requerimientos e intentar adaptarse rapidamente a las circunstancias cambiantes que van surgiendo a lo largo del desarrollo de un proyecto.

		\item De Scrum, las \textbf{stand up meetings}, que ayudan a los distintos miembros del equipo a conocer en que componentes trabajan sus pares y permite comprender el estado de avance de la iteración en detalle. Esto favorece la sincronización de los distintos miembros del equipo.

	\end{itemize}

\subsection{Comparación 'programming in the small' y 'programming in the large'}

	\newcommand{\pil}{
		\emph{Programming in the large}
	}

	\newcommand{\pis}{
		\emph{Programming in the small}
	}

	\pil y \pis se utilizan para describir distintos enfoques al desarrollo de software.

	Tipicamente el termino \pis se utiliza para describir el desarrollo de un sistema compuesto por una cantidad relativamente pequeña de subsistemas que interactuan entre si. Estos sistemas suelen desarrollarse por equipos pequeños en una cantidad corta de tiempo.

	Por otra parte, \pil hace referencia al desarrollo de sistemas, compuestos por varios subsistemas. De forma que el proyecto es realizado por un gran numero de equipos, o por un equipo en una cantidad grande de tiempo.

	De esta forma, podriamos pensar a \pil como el desarrollar grandes proyectos de software y a \pis como el desarrollo de cada uno de sus subcomponentes mas pequeños, con tareas simples y bien definidas.

	La principal diferencia que pudimos encontrar entre ambas, es la rigurosidad con la cual se planifica cada una. En \pis, la planificación resulta minima, esto se debe en particular a que se desarrollan modulos pequeños en un periodo corto de tiempo. En cambio en \pil, el diseño del sistema surge de un analisis mas profundo de la problematica. \pil no solo maneja proyectos mas grandes, sino que un error de diseño en un la arquitectura del sistema, puede requerir cambios en varios modulos, que requieran una gran cantidad de horas/hombre. Esto ultimo se contrasta con un error de diseño introducido en un unico modulo, donde los cambios estan aislados.

	\nota{Despues si alguien puede abultar un poco mas}
