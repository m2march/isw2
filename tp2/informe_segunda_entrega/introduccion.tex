Para este trabajo practico se nos pide planificar y diagramar la arquitectura de una aplicación para ahorradores. Esta aplicación, de forma similar a \emph{Precio Justo} (presentada en el primer trabajo practico), deberá servir para que ahorradores puedan consultar a través por ofertas, de distintos productos, publicadas en distintas redes sociales y medios de internet. Sin embargo, a diferencia de \emph{Precio Justo}, el alcance de esta aplicación resulta ser mucho mayor. 

Para esta entrega se nos pide:
\begin{enumerate}
	\item \emph{Atributos de calidad} identificadas en la situación planteada, presentado en la sección \ref{sec:calidad}.
	\item Diagrama de la \emph{arquitectura}, en la seccion \ref{sec:arquitectura}. Ademas, se explica como esta satisface los atributos de calidad, planteados en esta entrega, y los casos de uso introducidos en la entrega anterior.
	\item Comparación entre \emph{Unified Process} y \emph{Scrum}, y comparación entre \emph{Programming in the small} con \emph{Programming in the large}. Esto ultimo se detalla en la sección \ref{sec:comparacion}.
\end{enumerate}

Adicionalmente en la sección \ref{sec:cu}, los casos de uso de la entrega anterior son recordados.
