Para este trabajo práctico se nos pide planificar y diagramar la arquitectura de una aplicación para ahorradores. Esta aplicación, de forma similar a \emph{Precio Justo} (presentada en el primer trabajo práctico), deberá servir para que los ahorradores puedan consultar ofertas de distintos productos, publicadas en varias redes sociales o medios de Internet. Sin embargo, a diferencia de \emph{Precio Justo}, el alcance de esta aplicación resulta ser mucho mayor. 

Para esta entrega se nos pide:
\begin{enumerate}
	\item \emph{Atributos de calidad} identificados en la situación planteada, presentado en la sección \ref{sec:calidad}.
	\item Diagrama de la \emph{arquitectura}, en la sección \ref{sec:arquitectura}. Además, se explica cómo la arquitectura satisface los atributos de calidad, planteados en esta entrega, y los casos de uso introducidos en la entrega anterior.
	\item Comparación entre \emph{Unified Process} y \emph{Scrum}, y también una comparación entre \emph{Programming in the small} y \emph{Programming in the large}. Esto último se detalla en la sección \ref{sec:comparacion}.
\end{enumerate}

Adicionalmente, en la sección \ref{sec:cu}, los casos de uso de la entrega anterior son recordados.
