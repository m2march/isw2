\begin{itemize}
\item
  \emph{Extensibilidad} de las fuentes de datos. (2do párrafo enunciado)

  \begin{itemize}
  \item
    Fuente: El equipo de desarrolladores.
  \item
    Estimulo: Se desea implementar una nueva fuente de datos.
  \item
    Artefacto: Sistema de obtención de datos.
  \item
    Entorno: En funcionamiento normal.
  \item
    Respuesta: Se implementa la fuente de datos y se la integra al
    sistema.
  \item
    Medida: La fuente de datos se integra con el sistema en menos de 1
    hora, sin detener al sistema.
  \end{itemize}
\item
  \emph{Modificabilidad} de los bots que obtienen datos.

  \begin{itemize}
  \item
    Fuente: El equipo de desarrolladores.
  \item
    Estimulo: Se desea implementar/modificar/eliminar un bot para
    obtener datos de una pagina web determinada.
  \item
    Artefacto: Sistema de obtención de datos.
  \item
    Entorno: En funcionamiento normal.
  \item
    Respuesta: Se implementa el bot y se integra con el sistema.
  \item
    Medida: el bot se implementa o modifica en menos de 25 horas y se
    integra con el sistema en menos de 1 hora, sin detener al sistema.
  \end{itemize}
\item
  \emph{Detección de fallas} en la información obtenida por los bots.

  \begin{itemize}
  \item
    Fuente: Las paginas monitoreadas.
  \item
    Estimulo: Se produce un cambio en la estructura de las paginas
    monitoreadas.
  \item
    Artefacto: Bot del sistema de obtención de datos.
  \item
    Entorno: En funcionamiento normal.
  \item
    Respuesta: Se detecta el cambio en la estructura de la pagina, se
    detiene el bot, y se informa a los administradores.
  \item
    Medida: Cuando de una serie de consultas, se detecta un 70\% de
    cambios en su estructura, se considera como error.
  \end{itemize}
\item
  \emph{Modificabilidad} de los rubros y productos de rubros. (3pe)

  \begin{itemize}
  \item
    Fuente: Administradores del sistema.
  \item
    Estimulo: Se desea agregar un nuevo producto/rubro.
  \item
    Artefacto: Sistema central TPA.
  \item
    Entorno: Funcionamiento normal.
  \item
    Respuesta: Utilizando la interfaz de administración, se agrega un
    nuevo producto/rubro al sistema, sin detenerlo.
  \item
    Medida: Se agrega un producto/rubro en menos de 5 minutos.
  \end{itemize}
\item
  \emph{Modificabilidad} de las reglas de asociación y sustitución.
  (3pe)

  \begin{itemize}
  \item
    Fuente: Administradores del sistema
  \item
    Estimulo: Se desea agregar/modificar las reglas de asociación y
    sustitución.
  \item
    Artefacto: Sistema central TPA.
  \item
    Entorno: Modo de mantenimiento.
  \item
    Respuesta: Se modifican las reglas de asociación y sustitución.
  \item
    Medida: En menos de 24hs se implementan y ponen en funcionamiento
    las modificaciones a estas reglas.
  \end{itemize}
\item
  \emph{Performance} para velocidad en que se identifican ofertas de las
  distintas fuentes de datos. (4pe)

  \begin{itemize}
  \item
    Fuente: Usuario externo
  \item
    Estimulo: Se publica en un medio monitoreado por el sistema de
    obtención de datos una oferta.
  \item
    Artefacto: Sistema de obtención de datos.
  \item
    Entorno: Funcionamiento normal.
  \item
    Respuesta: El sistema de obtención de datos identifica esta oferta y
    la agrega a la base de datos.
  \item
    Medida: La oferta se comienza a tener en cuenta 10 segundos despues
    de que fue publicada por una fuente de datos.
  \end{itemize}
\item
  \emph{Usabilidad}, sistema de confianza facilmente configurable.

  \begin{itemize}
  \item
    Fuente: Usuario autentificado.
  \item
    Estimulo: Modifica las reglas de confianza de ofertas.
  \item
    Artefacto: Sistema central TPA.
  \item
    Entorno: Funcionamiento normal.
  \item
    Respuesta: Se modifican las reglas de confianza de ofertas.
  \item
    Medida: El sistema provee una guia de configuración de confianza de
    ofertas, que ayuda a completar dicha tarea en menos de 15 minutos
    para un usuario nuevo.
  \end{itemize}
\item
  \emph{Extensibilidad} de las fuentes de confianza de datos del
  usuario.

  \begin{itemize}
  \item
    Fuente: Equipo de desarrollo.
  \item
    Estimulo: Se desea agregar una nueva fuente de confianza de datos
    del usuario.
  \item
    Artefacto: Sistema central TPA.
  \item
    Entorno: Modo de mantenimiento.
  \item
    Respuesta: Se agrega la fuente de confianza de datos al sistema.
  \item
    Medida: La fuente de confianza se implementa en menos de 25hs y se
    integra al sistema en menos de 1 hora.
  \end{itemize}
\item
  \emph{Modificabilidad} del servicio de deteccion de spam.

  \begin{itemize}
  \item
    Fuente: Equipo de desarrollo.
  \item
    Estimulo: Se desea modificar el funcionamiento del servicio de
    detección de Spam.
  \item
    Artefacto: Sistema de detección de Spam.
  \item
    Entorno: Funcionamiento normal.
  \item
    Respuesta: Se modifica el funcionamiento del servicio.
  \item
    Medida: El funcionamiento del servicio de detección de Spam debe
    poder modificarse (Cambio de proveedor, nuevo proveedor,
    etc\ldots{}) con el sistema en funcionamiento Normal, y deberia ser
    posible que mas de un sistema de detección de Spam puedan
    co-existir.
  \end{itemize}
\item
  \emph{Auditabilidad} para ver la cantidad de ofertas falsas
  detectadas, productos de precios dudosos, etc.

  \begin{itemize}
  \item
    Fuente: Administradores del sistema.
  \item
    Estimulo: Se desea ver el resumen de ofertas falsas, detectadas,
    etc\ldots{}
  \item
    Artefacto: Sistema de detección de Spam.
  \item
    Entorno: Funcionamiento normal.
  \item
    Respuesta: Se obtiene un resumen con la información de ofertas
    falsas detectadas, precios dudosos, etc\ldots{}
  \item
    Medida: Cada vez que se elimina / modifica una oferta se registra
    `quien, cuando, por que (precio dudoso, oferta falsa, etc\ldots{}),
    y la oferta'.
  \end{itemize}
\item
  \emph{Usabilidad} de la interfaz del sistema para realizar consultas.

  \begin{itemize}
  \item
    Fuente: Un usuario nuevo.
  \item
    Estimulo: El usuario desea realizar una consulta.
  \item
    Artefacto: Interfaz Movil / Interfaz Web.
  \item
    Entorno: Funcionamiento normal.
  \item
    Respuesta: Se realiza la consulta y se informa de los resultados a
    los usuarios.
  \item
    Medida: En menos de 5 minutos, un usuario nuevo comprende la
    interfaz y comienza a utilizarla .
  \end{itemize}
\item
  \emph{Usabilidad} de la interfaz del sistema mientras se realizan
  consultas.

  \begin{itemize}
  \item
    Fuente: Un usuario.
  \item
    Estimulo: El usuario desea realizar una consulta.
  \item
    Artefacto: Interfaz Movil / Interfaz Web.
  \item
    Entorno: Funcionamiento normal.
  \item
    Respuesta: Conforme se escribe la respuesta, se muestran resultados.
  \item
    Medida: En menos de 1 segundo luego de que se comenzo a tipear una
    consulta, el sistema comienza a sugerir posibles resultados.
  \end{itemize}
\item
  \emph{Usabilidad} de la interfaz del sistema, antes de realizar
  consultas.

  \begin{itemize}
  \item
    Fuente: Un usuario.
  \item
    Estimulo: Se accede a la interfaz y aun no se realiza ningun tipo de
    consulta.
  \item
    Artefacto: Interfaz Movil / Interfaz Web.
  \item
    Entorno: Funcionamiento normal.
  \item
    Respuesta: Se comienzan a presentar resultados.
  \item
    Medida: Se muestran las ofertas populares / mas buscadas / mas
    recomendadas.
  \end{itemize}
\item
  \emph{Disponibilidad} del servicio.

  \begin{itemize}
  \item
    Fuente: Usuario.
  \item
    Estimulo: Se desea realizar una consulta.
  \item
    Artefacto: Interfaz movil.
  \item
    Entorno: Funcionamiento normal.
  \item
    Respuesta: Se realiza la consulta al sistema, se obtienen resultados
    y son mostrados al usuario.
  \item
    Medida: En el 99\% de los casos la consulta se realiza con exito al
    sistema central.
  \end{itemize}
\item
  \emph{Disponibilidad} del servicio cuando no hay conección.

  \begin{itemize}
  \item
    Fuente: Usuario.
  \item
    Estimulo: Se desea realizar una consulta.
  \item
    Artefacto: Interfaz movil.
  \item
    Entorno: Funcionamiento sin conección
  \item
    Respuesta: Si la consulta se encuentra disponible sin conección, se
    obtienen los resultados y son mostrados al usuario.
  \item
    Medida: Las ultimas consultas y las consultas mas populares al
    momento de la ultima conección se encuentran disponibles.
  \end{itemize}
\end{itemize}
