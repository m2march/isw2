\documentclass[10pt, a4paper]{article}

\usepackage[paper=a4paper, left=1.5cm, right=1.5cm, bottom=1.5cm, top=1.5cm]{geometry}
\usepackage[utf8]{inputenc}
\usepackage[spanish]{babel}
\usepackage{graphicx}
\usepackage{multicol}
\usepackage[usenames,dvipsnames]{color}
\usepackage{amsmath}
\usepackage{verbatim}
\usepackage{footnote}
\usepackage{float}
\usepackage{amsfonts}
\usepackage{hyperref}
\usepackage{framed}
\usepackage{pdflscape}

\usepackage{pdfpages}

\usepackage{caratula}


\materia{Ingeniería de Software II}

\titulo{Trabajo Práctico 1}
\subtitulo{Sprint Planning}

\integrante{Martín Alejandro Miguel}{181/09}{m2.march@gmail.com}
\integrante{Iván Postolski}{216/09}{ivan.postolski@gmail.com}
\integrante{Juan Manuel Martinez Caamaño}{276/09}{jmartinezcaamao@gmail.com}
\integrante{Matías Incem}{396/09}{matias.incem@gmail.com}
\integrante{Pablo Gauna}{334/09}{gaunapablo@gmail.com}

%%% Macros %%%

%userStory(idRally, titulo, descripcion, criterios-aceptacion, story-points, owner, iteración, estado)
\newcommand{\userStory}[8]{
	\fbox{
	\begin{minipage}{0.9\textwidth}

		\vspace{0.1cm}

		\begin{large}\textbf{#1: \textit{#2}}		\end{large}

		\vspace{11pt}

		\textbf{Descripción:} #3

		\vspace{9pt}

		\textbf{Criterios de aceptación:} #4

		\vspace{9pt}

		\begin{tabular}{ l c l }
		\textbf{Story Points:} #5 & \hspace{0.2\textwidth} & \textbf{Responsable:} #6 \\[6pt]
		\textbf{Iteración:} #7 & \hspace{0.2\textwidth} & \textbf{Estado:} #8 \\
		\end{tabular}

		\vspace{2pt}

	\end{minipage}
	}
}

%task(idRally, titulo, descripcion, horas-estimadas, responsable, horas-faltantes, estado)
\newcommand{\task}[7]{
	\fbox{
	\begin{minipage}{0.9\textwidth}

		\vspace{0.1cm}

		\begin{large}\textbf{#1: }\textit{#2}		\end{large}

		\vspace{5pt}

		\textbf{Descripción:} #3

		\vspace{3pt}

		\begin{tabular}{ l c l }
		\textbf{Horas estimadas:} #4 & \hspace{0.2\textwidth} & \textbf{Responsable:} #5 \\[1pt]
		\textbf{Horas faltantes:} #6 & \hspace{0.2\textwidth} & \textbf{Estado:} #7 \\
		\end{tabular}

		\vspace{2pt}

	\end{minipage}
	}
}

%%%%%%%%%%%%%%%


\begin{document}

\maketitle
\tableofcontents
\newpage

\section{Introducción}

El actual informe presenta el análisis, diseño y desarrollo inicial realizado para el producto \textbf{Precio Justo}. En este se detallan el conjunto de \emph{user stories} que abarcan el desarrollo completo de la aplicación, incluyendo tanto las definiciones actuales como la desiciones de extensión que se tienen en cuenta en este momento. Se presenta además un primer \emph{diseño orientado a objetos} de la aplicación completa. Este diseño tiene como intención estar abierto a los distintos ejes de cambios considerados en esta instancia. Por último, se analiza el alcance, el proceso y el éxito de la primera iteración. En esta sección se explica lo logrado y se generan conclusiones y propuestas de cambio para el resto del proceso de la aplicación.

\section{Análisis de la aplicación}

\subsection{Objetivo}

\textbf{Precio Justo} es una aplicación de recolección y resumen de datos masivos provistos por las redes sociales. La intención es poder obtener los precios más baratos para ciertos productos, de forma confiable e independiente. El uso de las redes sociales para obtener los datos tiene como objetivo lograr imparcialidad en los mismos.

\subsection{Ejes de cambio}

Dentro de la resolución genérica del objetivo de la aplicación se establecen, para una primera instancia, las siguientes restricciones, que pasan a conformar \emph{ejes de cambio} de la aplicación:

\paragraph{Obtención de datos}
\begin{itemize}
    \item \textbf{Origen de los datos:} En un primer lugar los datos a utilizar por la aplicación se obtendrán exclusivamente de \emph{twitter}. Dentro del flujo de datos del mismo, se espera filtrar aquellos que contengan información relevante para nuestros propósitos.
    \item \textbf{Formato de los tweets:} Considerando la restricción anterior, debe definirse bajo qué criterio un \emph{tweet} es útil. En una primera instancia se espera que los mismos tengan un formato similar a \textsf{$<$Producto$>$ $<$Precio$>$ $<$Unidad$>$ $<$Lugar$>$ \#PrecioJusto}, donde el \emph{hashtag} es indicador de que el \emph{tweet} está orientado a ser utilizado por nuestra aplicación. Se espera en un futuro poder precindir del \emph{hashtag} y poder interpretar formatos más liberales.
    \item \textbf{Unidad de los productos:} Otra restricción impuesta para el procesamiento de datos es que los precios estén indicados en su valor por unidad, siendo restringida las unidades aceptadas para cada producto. Por ejemplo, para tomates se aceptan solo kilogramos, y para aceite la unidad.
\end{itemize}

\section{User Stories}

%%%%% MOLDES PARA COMPLETAR STORIES Y TASKS

% \userStory	{} % ID en el Rally
% 			{} % título
% 			{} % descripcion
% 			{} % criterios de aceptacion
% 			{} % story points
% 			{} % story owner
% 			{} % iteración (Primera o ``no definida'')
% 			{} % estado (completada, bloqueada, etc.)

% \task	{} % ID en el Rally
% 		{} % título
% 		{} % descripcion
% 		{} % horas estimadas
% 		{} % responsable
% 		{} % horas faltantes
% 		{} % estado (completada, bloqueada, etc.)

\newpage

\subsection{Ejemplos de tareas}

\userStory	{US143} % ID en el Rally{Como desarrollador quiero un mecanismo para hacer tests unitarios} % título
			{Como desarrollador quiero un mecanismo para hacer tests unitarios} % título
			{Esta story se considera terminada cuando se encuentre implementado un mecanismo de fácil extensión para poder testear funcionalidades unitarias. } % descripcion
			{El mecanismo debe haber sido testeado en su facilidad de uso y extensibilidad creando tests.} % criterios de aceptacion
			{3} % story points
			{Martín M} % story owner
			{Primera} % iteración (Primera o ``no definida'')
			{En proceso} % estado (completada, bloqueada, etc.)

\task	{TA35} % ID en el Rally
		{Investigar como funcionan las apis (Rest y Streaming) junto con las sdk de twitter} % título
		{No había ninguna descripción escrita} % descripcion
		{4} % horas estimadas
		{Iván P} % responsable
		{0} % horas faltantes
		{Completada} % estado (completada, bloqueada, etc.)

\task	{TA35} % ID en el Rally
		{Investigar como funcionan las apis (Rest y Streaming) junto con las sdk de twitter} % título
		{No había ninguna descripción escrita} % descripcion
		{4} % horas estimadas
		{Iván P} % responsable
		{0} % horas faltantes
		{Completada} % estado (completada, bloqueada, etc.)

\vspace{20pt}

\userStory	{US143} % ID en el Rally{Como desarrollador quiero un mecanismo para hacer tests unitarios} % título
			{Como desarrollador quiero un mecanismo para hacer tests unitarios} % título
			{Esta story se considera terminada cuando se encuentre implementado un mecanismo de fácil extensión para poder testear funcionalidades unitarias. } % descripcion
			{El mecanismo debe haber sido testeado en su facilidad de uso y extensibilidad creando tests.} % criterios de aceptacion
			{3} % story points
			{Martín M} % story owner
			{Primera} % iteración (Primera o ``no definida'')
			{En proceso} % estado (completada, bloqueada, etc.)

\task	{TA35} % ID en el Rally
		{Investigar como funcionan las apis (Rest y Streaming) junto con las sdk de twitter} % título
		{No había ninguna descripción escrita} % descripcion
		{4} % horas estimadas
		{Iván P} % responsable
		{0} % horas faltantes
		{Completada} % estado (completada, bloqueada, etc.)

\section{Diseño}

\section{Primera Iteración}

\subsection{Desarrollo}

\subsection{Logros}

\subsection{Review}

\end{document}
