
%%% Macros %%%

%userStory(idRally, titulo, descripcion, criterios-aceptacion, story-points, owner, iteración, estado)
\newcommand{\userStory}[8]{
	\fbox{
	\begin{minipage}{0.9\textwidth}
    {\small \sffamily

            %\vspace{0.1cm}

            {\normalsize \textbf{#1:} \textit{#2}}

            \vspace{3pt}

            \textbf{Descripción:} #3

            \vspace{1pt}

            \textbf{Criterios de aceptación:} #4


%            \begin{tabular}{ l c l }
%            \textbf{Story Points:} #5 & \hspace{0.2\textwidth} & \textbf{Responsable:} #6 \\[6pt]
%            \textbf{Iteración:} #7 & \hspace{0.2\textwidth} & \textbf{Estado:} #8 \\
%            \end{tabular}
            \begin{center}
                    \begin{tabular}{ l l l }
                    \textbf{Story Points:} #5 & \textbf{Iteración:} #7  & \textbf{Estado:} #8 \\
                    \end{tabular}
            \end{center}
    }
	\end{minipage}
	}
}

%task(idRally, titulo, descripcion, horas-estimadas, responsable, horas-faltantes, estado)
\newcommand{\task}[7]{
  \hspace{7pt}
	\fbox{
	\begin{minipage}{0.88\textwidth}
    {\small \sffamily

            %\vspace{0.1cm}

            {\normalsize \textbf{#1:} \textit{#2}}

            \vspace{3pt}

            \textbf{Descripción:} #3

            \vspace{5pt}

            \begin{tabular}{ l c l }
            \textbf{Horas estimadas:} #4 & \hspace{0.2\textwidth} & \textbf{Responsable:} #5 \\[1pt]
            \textbf{Horas faltantes:} #6 & \hspace{0.2\textwidth} & \textbf{Estado:} #7 \\
            \end{tabular}

            %\vspace{2pt}
    }
	\end{minipage}
	}
}

%%%%%%%%%%%%%%%

%%%%% MOLDES PARA COMPLETAR STORIES Y TASKS

% \userStory	{} % ID en el Rally
% 			{} % título
% 			{} % descripcion
% 			{} % criterios de aceptacion
% 			{} % story points
% 			{} % story owner
% 			{} % iteración (Primera o ``no definida'')
% 			{} % estado (completada, bloqueada, etc.)

% \task	{} % ID en el Rally
% 		{} % título
% 		{} % descripcion
% 		{} % horas estimadas
% 		{} % responsable
% 		{} % horas faltantes
% 		{} % estado (completada, bloqueada, etc.)

\newpage

\subsection{Ejemplos de tareas}

\userStory	{US143} % ID en el Rally{Como desarrollador quiero un mecanismo para hacer tests unitarios} % título
			{Como desarrollador quiero un mecanismo para hacer tests unitarios} % título
			{Esta story se considera terminada cuando se encuentre implementado un mecanismo de fácil extensión para poder testear funcionalidades unitarias. } % descripcion
			{El mecanismo debe haber sido testeado en su facilidad de uso y extensibilidad creando tests.} % criterios de aceptacion
			{3} % story points
			{Martín M} % story owner
			{Primera} % iteración (Primera o ``no definida'')
			{En proceso} % estado (completada, bloqueada, etc.)

\task	{TA35} % ID en el Rally
		{Investigar como funcionan las apis (Rest y Streaming) junto con las sdk de twitter} % título
		{No había ninguna descripción escrita} % descripcion
		{4} % horas estimadas
		{Iván P} % responsable
		{0} % horas faltantes
		{Completada} % estado (completada, bloqueada, etc.)

\task	{TA35} % ID en el Rally
		{Investigar como funcionan las apis (Rest y Streaming) junto con las sdk de twitter} % título
		{No había ninguna descripción escrita} % descripcion
		{4} % horas estimadas
		{Iván P} % responsable
		{0} % horas faltantes
		{Completada} % estado (completada, bloqueada, etc.)

\vspace{20pt}

\userStory	{US143} % ID en el Rally{Como desarrollador quiero un mecanismo para hacer tests unitarios} % título
			{Como desarrollador quiero un mecanismo para hacer tests unitarios} % título
			{Esta story se considera terminada cuando se encuentre implementado un mecanismo de fácil extensión para poder testear funcionalidades unitarias. } % descripcion
			{El mecanismo debe haber sido testeado en su facilidad de uso y extensibilidad creando tests.} % criterios de aceptacion
			{3} % story points
			{Martín M} % story owner
			{Primera} % iteración (Primera o ``no definida'')
			{En proceso} % estado (completada, bloqueada, etc.)

\task	{TA35} % ID en el Rally
		{Investigar como funcionan las apis (Rest y Streaming) junto con las sdk de twitter} % título
		{No había ninguna descripción escrita} % descripcion
		{4} % horas estimadas
		{Iván P} % responsable
		{0} % horas faltantes
		{Completada} % estado (completada, bloqueada, etc.)
